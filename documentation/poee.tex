\documentclass[12pt]{article}

\usepackage[utf8]{inputenc}
\usepackage{latexsym,amsfonts,amssymb,amsthm,amsmath}
\usepackage[T1, T2A]{fontenc}% T2A for Cyrillic font encoding
\usepackage[bulgarian]{babel}
\usepackage{float}
\usepackage{graphicx}
\usepackage{amssymb}
\usepackage{emoji}
\usepackage{hyperref}
\usepackage[
backend=biber,
style=alphabetic,
]{biblatex}
\newcommand{\cyrchar}[1]{\foreignlanguage{bulgarian}{#1}}


\graphicspath{ {./images/} }


\setlength{\parindent}{0in}
\setlength{\oddsidemargin}{0in}
\setlength{\textwidth}{6.5in}
\setlength{\textheight}{8.8in}
\setlength{\topmargin}{0in}
\setlength{\headheight}{18pt}

\begin{document}
	
	\title{ Подходи за обработка на естествен език \newline SemEval-2025 Task-3 — Mu-SHROOM  \newline \hline}
	
	
	\author{Кристиян Симов, ф.н. 4MI3400288 \\ Цветан Цветанов, ф.н. 4MI3400570 \\ Изкуствен интелект}
	\maketitle
	
	% \begin{Logo}
		\begin{figure}[H]
			\centering
			\includegraphics[width=0.25\linewidth]{clement-ohrid-logo.png}
		\end{figure}
		
		\begin{figure}[H]
			\centering
			\includegraphics[width=0.25\linewidth]{fmi-logo.jpg}
		\end{figure}
		% \end{Logo}
	
	
	\vspace{0.5in}
	\pagebreak
	
	\section{Въведение}
	
	Езиковите модели са вещи в естествените езици и звучат убедително, но понякога генерират грешни и лъжовни твърдения, които нямат реална подкрепа с факти. 

	Халюцинирането е един от големите все още неразрешени проблеми  на големите езикови модели.  Задачата  Mu-SHROOM в изданието на SemEval от 2025 е продължение на SHROOM от предишното издание. През 2024 година задачата пред участниците е била да класифицират дали даден текст е халюцинация (да или не). Промяната в сегашното издание е, че се очаква да се предскаже началото и края на халюцинация в изходния текст на конкретен модел.
	

	\section{Данни}
	
	Използвани са единствено публично достъпни данни от страницата на задачата. Не е правен опит да се генерират синтетични данни. Организаторите са подготвили текстови данни за трениране, тестване и валидация на 14 различни езика в jsonl формат. 
	
	\section{Метод}
	
	\subsection{Обработка на данните}
	
	\subsection{Алгоритми}
	
	\section{Експерименти}
	
	\section{Резултати}
	
	\section{Заключение и бъдеща работа}
	
	\pagebreak
	\section{Индивидуален принос}
	
	
\end{document}